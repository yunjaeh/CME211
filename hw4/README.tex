\documentclass{article}
\usepackage{graphicx}

\title{CME211:HW4 writeup}
\author{Yunjae Hwang}

\begin{document}
\maketitle


\section*{Code structure}
The "Truss" class consists of four methods including \_\_init\_\_ and \_\_repr\_\_.
To run this class, beams and joints data files should be provided.
And optionally, it will save the figure of shape of geometry if the output figure name is provided.

In the first method, \_\_init\_\_ is for reading data and initializing variables that will be used in the class.
It initializes "A" matrix, and "x" and "b" vectors using csr\_matrix(one type of sparse matrix) in the Scipy.
As the structures given in the homework have a small number of beams, 
memory usage is not significantly different between dense matrix and sparse matrix.
However, the more elements a truss has, the more memory usage and computational cost will be required.
So the use of sparse matrix will be more effective in larger size of system.

The second method(\_\_repr\_\_) is to print the summary of forces in beams.
Positive values represent tension in beams and negative forces mean compression in beams, respectively.


Third method names "PlotGeometry" and it saves the shape geometry of given geometry input.
This is optional and the method runs once the file name of the figure is given.
The file name should be given as a input of this method.
One example is shown in Figure \ref{fig:truss}.


Lastly, "ComputeStaticEquilibrium" is the most important one.
Which builds a system of equations( $Ax=b$ ) and solve the system using linear sparse solver.
No specific inputs are required.
It handles errors in solving the system such as singular matrix and non-square matrix.


\section*{Error handling}
As illustrated in the instruction, truss 1 and 2 are suitable for static equilibrium analysis,
while truss 3 and 4 are not. The "A" matrix of the last two cases is either non-square or singular.
This is due to the system is phyiscally unstable and additional assumptions are needed for having 
well-posed system of constitutive equtions.
My program handles these cases as an exception and prints out an error message.
\begin{itemize}
\item For non-square matrix A, before calling the "ComputeStaticEquilibrium", the program raise a RuntimeError.
\item For singular matrix A, sparse linear solver actually computes the equations and prints warning.
The warning is also considered as an error by using imported "warnings" module.
\end{itemize}


\begin{figure}
    \centering
    \includegraphics[width=0.8\textwidth]{truss1.png}
    \caption{\label{fig:truss}Sample figure of truss geometry : Truss 1}
\end{figure}

\end{document}
