\documentclass{article}
\usepackage[]{algorithm}
\usepackage[]{algorithmic}

\title{CME211:Project Part1 writeup}
\author{Yunjae Hwang}

\begin{document}
\maketitle

The steps below are the order of commands running in main.cpp.
Since the commands in the program runs sequentially and not complicated, 
it is visulized without using pseudo-code.
\begin{enumerate}
    \item Initialize vectors required for solving system of linear equations.
    \item Read sparse matrix data in COO format from a input file.
    \item Transform the data format (from COO to CSR).
    \item Call the function solves the system of equations 
    using the conjugate gradient(CG) algorithm.
    \item Print output (solution vector $\mathbf{x}$) to the output file.
\end{enumerate}

The main task for the first part of the project is 
to implement CG algorithm in the fourth step.
The algorithm below(Algorithm \ref{alg1}) summarizes 
the CG algorithm developed in my program.
Basically the same with the description 
in the assignment instruction.
First it has an intial guess of the solution vector 
$\mathbf{u}_0$ with all elements are 1.0.
It computes residual vector $\mathbf{r}_0$ 
and its second norm, accordingly.

After the initilaliztion, while loop starts and interates 
until either the ratio of the two residuals 
becomes less than the tolerance 
($ \|\mathbf{r}_{n+1}\|_2 / \|\mathbf{r}_0\|_2 < \epsilon $) or
the number of iteration exceeds the specified number of iteration 
($n_{iter} > n_{iter,max}$).
Once the while loop finishes, 
the $u_{n+1}$ vector will be returned to main function as the solution $x$.

Implementing "matvecops" significantly reduces the complexity 
in the calculation between a vector and a matrix.

In addition to this for reducing redundant programming, 
two functions are developed for computing the second norm of a vector 
and inner product of two vectors.


\begin{algorithm} 
    \caption{Conjugate Gradient (CG) algorithm 
    for solving a linear system: $\mathbf{Ax=b}$}
    \label{alg1}
    \begin{algorithmic}
        \STATE Initialize $\mathbf{u_0}$
        \STATE $\mathbf{r}_0 = \mathbf{b} - \mathbf{A} \mathbf{u}_0$
        \STATE compute $\|\mathbf{r}_0\|_2$
        \STATE $\mathbf{p}_0 = \mathbf{r}_0$
        
        \WHILE{$n_{iter} < n_{iter,max}$}
                \STATE $n_{iter} = n_{iter}+1$
                \STATE $\alpha = (\mathbf{r}_n^T \mathbf{r}_n)
                /(\mathbf{p}_n^T \mathbf{A} \mathbf{p}_n)$
                \STATE $\mathbf{u}_{n+1}  
                = \mathbf{u}_n + \alpha_n \mathbf{p}_n$
                \STATE $\mathbf{r}_{n+1}  
                = \mathbf{r}_n + \alpha_n \mathbf{A} \mathbf{p}_n$
                            
                \IF{$ \|\mathbf{r}_{n+1}\|_2 / \|\mathbf{r}_0\|_2 < \epsilon $}
                        \STATE \bf{break}
                \ENDIF
                \STATE $\beta_n 
                = (\mathbf{r}_{n+1}^T \mathbf{r}_{n+1})
                /(\mathbf{r}_n^T \mathbf{r}_n)$
                \STATE $\mathbf{p}_{n+1} 
                = \mathbf{r}_{n+1} + \beta_n \mathbf{p}_n $
        \ENDWHILE
        \STATE $\mathbf{x} = \mathbf{u}_{n+1}$
    \end{algorithmic}
\end{algorithm}

\end{document}